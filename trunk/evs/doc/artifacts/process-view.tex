\section{Process View}

The communication is based on Java sockets where each communication end point is identified by its IP address and TCP port number.

\subsection{Request Handlers}

The request handlers support two different communication protocols.

\subsubsection{Socket}

For each request the client request handler creates a new connection to the server request handler.
Every message is prefixed by a message header, which contains the length of the serialized request and the invocation style.
After the message header the message body is sent, which consists of the serialized request.
If the invocation style is fire and forget, then the client closes the socket without waiting for a response from the server.
Otherwise the client request handler waits to receive the message header of the response message.
Finally the client request handler receives the serialized response and returns it to the requestor.
\\
\\
For each request the server request handler waits to receive the message header and the serialized request.
The request is then passed to the invocation dispatcher, which returns the serialized response.
If the invocation style is fire and forget, then the server closes the socket without sending a response.
Otherwise the server sends a message header, which contains the length of the serialized response.
Finally the server request handler sends the serialized response.

\subsubsection{Axis}

\subsection{Message Patterns}

The request handlers support two different message patterns.

\subsubsection{Request Response}

The client request handler sends a serialized request to the server request handler.

\subsubsection{One Way}

\subsection{Invocation Styles}

The requestors support one synchronous and three asynchronous invocation styles.

\subsubsection{Synchronous}

\subsubsection{Poll Object}

\subsubsection{Result Callback}

\subsubsection{Fire and Forget}
